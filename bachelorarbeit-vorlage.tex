% Vorlage für eine Bachelorarbeit
% Siehe auch LaTeX-Kurs von Mathematik-Online
% www.mathematik-online.org/kurse
% Anpassungen für die Fakultät für Mathematik
% am KIT durch Klaus Spitzmüller und Roland Schnaubelt Dezember 2011

\documentclass[12pt,a4paper]{scrartcl}
% scrartcl ist eine abgeleitete Artikel-Klasse im Koma-Skript
% zur Kontrolle des Umbruchs Klassenoption draft verwenden


% die folgenden Packete erlauben den Gebrauch von Umlauten und ß
% in der Latex Datei
\usepackage[utf8]{inputenc}

%\usepackage[latin1]{inputenc} %  Alternativ unter Windows
\usepackage[T1]{fontenc}
%\usepackage[ngerman]{babel}
\usepackage{parskip}

\usepackage{nicefrac}


\usepackage[pdftex]{graphicx}
\usepackage{latexsym}
\usepackage{amsmath,amssymb,amsthm}
\usepackage[disable]{todonotes}

% Abstand obere Blattkante zur Kopfzeile ist 2.54cm - 15mm
\setlength{\topmargin}{-15mm}


% Umgebungen für Definitionen, Sätze, usw.
% Es werden Sätze, Definitionen etc innerhalb einer Section mit
% 1.1, 1.2 etc durchnummeriert, ebenso die Gleichungen mit (1.1), (1.2) ..
\newtheorem{Theorem}{Theorem}[section]
\newtheorem{Definition}[Theorem]{Definition} 
\newtheorem{Lemma}[Theorem]{Lemma}	
\newtheorem{Example}[Theorem]{Example}
\newtheorem{Remark}[Theorem]{Remark} 	   
                  
\numberwithin{equation}{section} 

% einige Abkuerzungen
\newcommand{\C}{\mathbb{C}} % komplexe
\newcommand{\K}{\mathbb{K}} % komplexe
\newcommand{\R}{\mathbb{R}} % reelle
\newcommand{\Q}{\mathbb{Q}} % rationale
\newcommand{\Z}{\mathbb{Z}} % ganze
\newcommand{\N}{\mathbb{N}} % natuerliche
\newcommand{\2}{\mathbb{Z} / 2 \mathbb{Z}}
\newcommand{\G}{\mathcal{G}}
\newcommand{\1}{\overline{1}}
\newcommand{\0}{\overline{0}}


\renewcommand{\labelenumi}{\roman{enumi})}



\begin{document}
  % Keine Seitenzahlen im Vorspann
  \pagestyle{empty}

  % Titelblatt der Arbeit
  \begin{titlepage}

    \includegraphics[scale=0.45]{kit-logo.jpg} 
    \vspace*{2cm} 

 \begin{center} \large 
    
    Masterarbeit
    \vspace*{2cm}

    {\huge Turing machines and irrational values of $l^2$-Betti numbers}
    \vspace*{2.5cm}

    Jan Kohlmüller
    \vspace*{1.5cm}

    20.08.2017
    \vspace*{4cm}


    Betreuung: Roman Sauer \\[1cm]
    Fakultät für Mathematik \\[1cm]
		Karlsruher Institut für Technologie
  \end{center}
\end{titlepage}



  % Inhaltsverzeichnis
  \tableofcontents

\newpage
 


  % Ab sofort Seitenzahlen in der Kopfzeile anzeigen
  \pagestyle{headings}

\section{Introduction}


\section{$l^2$-Betti numbers}
In this whole chapter $G$ is always a discrete countable group with multiplication $\cdot_G$.
\subsection{Von Neumann Dimension}

\begin{Definition} \label{GR}
	The group ring $RG$ over a Ring $R$ is the free module over $G$ with the additional multiplication
	\begin{align*}
		\sum_{h \in H} \alpha_h \cdot h \cdot_{RG} \sum_{k \in K} \beta_k \cdot k = \sum_{h \in H} \sum_{k \in K} \alpha_h \beta_k \cdot h \cdot_G k
	\end{align*}
	with $\alpha_h, \beta_k \in R$ and $H, K \in G$ finite subsets.
\end{Definition}
\begin{Definition}
	$l^2(G)$ is the Hilbert space consisting of formal sums $\sum_{g \in G} \lambda_g \cdot g$ such that $\lambda_g \in \C$ and $\sum_{g \in G} |\lambda_g|^2 < \infty$. The multiplication is defined as in \ref{GR} and the scalar product is defined through
	\begin{align*}
		\langle \sum_{g \in G} \alpha_g \cdot g, \sum_{g \in G} \beta_g \cdot g \rangle := \sum_{g \in G} \alpha_g \overline{\beta_g}.
	\end{align*}
\end{Definition}
\todo{hier hab ich nicht gezeigt, dass es ein Skalarprodukt ist}
\begin{Remark}
	If $G$ is a set instead of a group we can also define a Hilbert space $l^2(G)$. Of course we don not have a multiplication in this case.
\end{Remark}
\begin{Definition}
	The group von Neumann algebra $\mathcal{N}(G)$ is defined as the algebra of $G$-equivariant bounded operators from $l^2(G)$ to $l^2(G)$ where $G$ acts on $l^2(G)$ by left multiplication.
\end{Definition}
We will denote the bounded operators of a Hilbert space $H_1$ with $\mathcal{B}(H_1)$. If $G$ acts on a Hilbert Space $H_2$ we will call the subspace of $G$-equivariant elements of $H_2$ by $H_2^G$. Therefore we can also define the group von Neumann algebra through $\mathcal{N}(G) := \mathcal{B}(l^2(G))^G$.
\begin{Definition}
	Let $e_G \in G$ be the unit element of $G$. Then
	\begin{align*}
		tr_{\mathcal{N}(G)}: \mathcal{N}(G) \to \C, A  \mapsto \langle A(1 \cdot e_G), 1 \cdot e_G \rangle
	\end{align*}
	is called the von Neumann trace on $\mathcal{N}(G)$.
\end{Definition}
\begin{Remark}
	Right multiplication with an element of the Group ring $\C G$ is a $G$-equivariant operator on $l^2(G)$ and therefore has a well defined von Neumann trace.
\end{Remark}
\begin{Definition}
	A Hilbert $\mathcal{N}(G)$-module is a Hilbert space $H$ with a linear isometric $G$-action such that there exists an isometric linear $G$-embedding from $H$ to $l^2(G)^n$ for some $n \in \N$.
\end{Definition}
\begin{Definition}
	Let $H$ be a Hilbert $\mathcal{N}(G)$-module with corresponding embedding $\iota: H \to l^2(G)^n$ and let $pr_{\iota(H)}: l^2(G)^n \to H$ be the corresponding $G$-equivariant projection. Let $\iota_i: l^2(G) \to l^2(G)^n, y \mapsto (x_1, \ldots, x_n)$ with $x_j = \delta_{i,j} \cdot y$ the inclusion of the i-th $l^2(G)$ and $pr_i: l^2(G)^n \to l^2(G), (x_1, \ldots, x_n) \mapsto x_i$ the corresponding projection. Then 
	\begin{align*}
		tr_{\mathcal{N}(G)}: \mathcal{B}(H)^G \to \C, A \mapsto \sum_{i = 1}^{n} tr_{\mathcal{N}(G)}(pr_i \circ \iota \circ A \circ pr_{\iota(H)} \circ \iota_i)
	\end{align*}
	is called the von Neumann trace on $H$.
\end{Definition}
One can show that the trace is independent of the chosen embedding and therefore well-defined. \todo{eine oder mehrere Projektionen?} 
\begin{Definition}
	Let $H$ be a Hilbert $\mathcal{N}(G)$-module. We define the von Neumann dimension of $H$ as
	\begin{align*}
		dim_{\mathcal{N}(G)}(H) = tr_{\mathcal{N}(G)}(id_H).
	\end{align*}
\end{Definition}

\subsection{$l^2$-Betti numbers of $G$-CW-complexes}
In this chapter we will require some basic knowledge of CW-complexes and regular homology. For more information on this topic see e.g.\cite{HATCH}.
\begin{Definition}
	Let $X$ be a CW-complex and $\rho: G \to Homeo(X)$ an action of $G$ on $X$ by homeomorphism such that for every open cell $E \subset X$ and every $g \in G$ it holds that $\rho(g)(E)$ is again an open cell and $\rho(g)|_{E} = id_E$ if $\rho(g)(E) \cap E \neq \emptyset$. Then $\rho$ is called a cellular action.
\end{Definition}

\begin{Definition}
	A $G$-CW-complex is a CW-complex together with a cellular $G$-action.
\end{Definition}

\begin{Definition}
	The orbit of an $n$-cell in a $G$-CW-complex is called a $G$-equivariant $n$-cell.
\end{Definition}

\begin{Definition}
	Let $X$ be a $G$-CW complex. $X$ is called
	\begin{itemize}
		\item proper if all stabilizer groups are finite,
		\item free if all stabilizer groups are trivial,
		\item finite type if for every $n \in \N$ it has only finitely many $G$-equivariant $n$-cell.
	\end{itemize}
\end{Definition}
\begin{Remark}
	One can show that the chain complex of a $G$-CW complex $X$ is a chain complex $C_*(X)$ of left $\Z G$-modules, i.e. each $C_n(X)$ is a left $\Z G$-module and the differentials are homomorphisms of these modules.
\end{Remark}
\begin{Definition}
	The $l^2$-chain complex of a $G$-CW complex $X$ is
	\begin{align*}
		C_*^{(2)}(X) = l^2G \otimes_{\Z G} C_*(X).
	\end{align*}
\end{Definition}
\todo{Hier muss man die Definition evtl noch an das richtige Cstar anpassen, l2G ist rechtes ZG modul, was heißt es an chain complex zu tensoren, der Abschluss?}

\begin{Definition}
	Let $X$ be a proper, finite type $G$-CW complex and $C_*^{(2)}(X)$ the corresponding $l^2$-chain complex with its differential $d_*^{(2)}$. Then 
	\begin{align*}
		H_n^{(2)}(X) := ker(d_n^{(2)}) / \overline{im(d_{n+1}^{(2)})}
	\end{align*}
	is called the $n$-th $l^2$-homology and 
	\begin{align*}
		b_n^{(2)}(X)=dim_{\mathcal{N}(G)}H_n^{(2)}(X)
	\end{align*}
	the $n$-th $l^2$-Betti number.
\end{Definition}
\begin{Remark}
	One can show that for a proper, finite type $G$-CW complex $H_n^{(2)}(X)$ is a Hilbert $\mathcal{N}(G)$-module and therefore the $l^2$-Betti number is in fact well-defined.
\end{Remark}

\subsection{$l^2$-Betti numbers arising from a group}
\begin{Remark}\label{MAB}
	A Matrix $A \in \C G^{n \times m}$ can be seen as a map from $l^2G^n \to l^2G^m$ via $x \mapsto x^{\top} \cdot A$. Therefore $ker(A)$ is a Hilbert $\mathcal{N}(G)$-module and has a well-defined von-Neumann-dimension. 
\end{Remark}
\begin{Theorem} \label{MCW}
	Let $x \in \R$ and $G$ be a discrete, countable, finitely generated group. The following are equivalent:
	\begin{enumerate}
		\item There exists a cocompact free finite type $G$-CW-complex $X$ and a $n \in \N$ such that $b_n^{(2)}(X)=x$
		\item There exists a Matrix $A \in \Q G^{n \times n}$ for a natural numbers $n \in \N$ such that \newline $dim_{\mathcal{N}(G)}(ker (A))=x$
	\end{enumerate}
\end{Theorem}
\begin{proof}
	$"i) \Rightarrow ii)"$ Let $X$ be a cocompact free finite type $G$-CW-complex and $n \in \N$ such that $b_n^{(2)}(X)=x$. We define 
	\begin{align*}
		\Delta_n^{(2)} = d_n^{(2)*} d_n^{(2)} + d_{n+1}^{(2)} d_{n+1}^{(2)*},
	\end{align*}
	which is a map from $C_n^{(2)}(X)$ to $C_n^{(2)}(X)$ and for which $ker(\Delta_n^{(2)}) = H_n^{(2)}$ holds. With \ref{MAB} $ii)$ follows. \todo{Warum ist das ne Matrix mit Eintraegen aus Q?}
	
	$"ii) \Rightarrow i)"$ Let $n \in \N$ and $A\in \Q G^{n \times n}$. Because $k \cdot A$ has the same kernel as $A$ for every $k \in \Z$ we can assume that $A \in \Z G^{n \times n}$. The corresponding map will be also called $A$. We will construct a $G$-CW-complex $X$ with $d_3^{(2)} = A$ and $d_4^{(2)}$ trivial. Therefore $H_3^{(2)}(X) = ker(A)$ and $i)$ follows.
	
	First we can see that a graph can always be seen as a CW-complex with the edges as 1-cells and the vertices as 0-cells. Let $g_1 \ldots g_m$ be generators of $G$ and let $Y$ be the corresponding Cayley graph of $G$ which has a natural cellular free $G$-action and is therefore a free, finite type $G$-CW-complex. In addition it has exactly $1$ $G$-equivariant $0$-cell and $m$ $G$-equivariant $1$-cell and no other cells. We then glue $n$ many $2$-cells on each $0$-cell, i.e. we attach them via the attaching map which sends all of $S^1$ to a single point. 
	
	We now glue in $n$ many $G$-equivariant $3$-cells in such a way that $A$ corresponds to $d_3^{(2)}$. Let $A_{ij} = \sum_{g \in G} z_{(g, i, j)} g \in \Z G$ denote the entry in the i-th row and the j-th column of $A$. We fix one $0$-cell and call it $e$ and we call the corresponding $2$-cells $e_1 \ldots e_n$. We glue in $G \times D^3$ where each element $h \times D^3$ is glued $z_{(g, 1, j)}$ many times to $h g \cdot e_j$ for each $g \in G$ and $j \in \{1, \ldots, n\}$. We then see that $d_3^{(2)}$ maps the vector $(\sum_{g \in G} \lambda_g g, 0 , \ldots , 0) \in (l^2 G)^n$ to $(\sum_{g \in G} \sum_{h \in H} \lambda_g z_{(h, 1, 1)} g h, \ldots , \sum_{g \in G} \sum_{h \in H} \lambda_g z_{(h, 1, n)}) \in (l^2 G)^n$. We then repeat this for every row of $A$ and get that $d_3^{(2)} = A$. The resulting $G$-CW-complex $X$ is still free, finite type and because $X/G$ is finite $X$ is cocompact.
\end{proof}
\todo{ker(delta) = homology zeigen?}
\begin{Remark}
	One can show that \ref{MCW} also holds if $G$ is not finitely generated.
\end{Remark}
\begin{Definition}
	We say that a number $x \in \R$ arises from a discrete countable group $G$ if it fulfills one of the above conditions.
\end{Definition}
\begin{Lemma}\label{add}
	If $x \in \R$ and $z \in \R$ both arise from a discrete, countable Group $G$, then so does $x + z$.
\end{Lemma}
\begin{Lemma}\label{mult}
	Let $G$ is a discrete countable Group and $H$ a discrete finite Group. If $x \in \R$ arises from $G \times H$ then $x \cdot |H|$ arises from $G$.
\end{Lemma}
\todo{Vllt. noch beweisen}
\section{Turing dynamical systems}
%Hier definiere ich TDS
Let us first give a definition of a Turing machine. For convenience we will only allow 1 and 0 as symbols on our tape.
\begin{Definition}
	A Turing machine is a 5-tuple $T=(S,\delta, A, R, I)$ where S is a finite set of states and $\delta: S \setminus(A \cup R) \times \2 \to S \times \2 \times \{-1, 0, 1\}$ is called the transition function. $A \subset S$ is called the set of accepting states, $R \subset S$ is called the set of rejecting states and $I \in S$ is called the initial state.
\end{Definition}
In addition each Turing machine uses a "tape", i.e. a set $\Z / 2\Z ^{\Z}$ with a "head" on the element corresponding to the index 0. A Turing machine can operate on an element of $Y \in \Z / 2\Z ^{\Z}$ which is called an input. We start with the initial state and use the transition function $\delta$ to get $\delta(I, Y_0)$. The first coordinate corresponds to the new element $Y_0$, the second to the new state and the third corresponds to shifting the tape to the left or the right and therefore getting a new $Y_0$, e.g. if $\delta(I, Y_0)_3 = 1$ we shift the whole tape one to the left, therefore our new $Y_0$ is our old $Y_1$. We repeat this step until we get a state in $A \cup R$. We say, that the Turing machine accepts $Y$ if we get a state in $A$ after finitely many steps and it rejects $Y$ if we get a state in $R$ after finitely many steps. We say that the Turing machine holds for $Y$ if it accepts or rejects it.

For the purpose of calculating $l^2$-Betti numbers of groups we need to extend this definition. Let $(X, \mu)$ be a probability measure space divided into finitely many disjoint measurable subsets $X_i$. Let $\Gamma$ be a countable discrete group and $\rho$ be a right measure preserving action of $\Gamma$ on X.
We now choose 3 disjoint subsets $A$, $R$, $I$, where each of them is a union of certain $X_i$. They will be called the accepting set, the rejecting set and the initial set. In addition we choose a $\gamma_i \in \Gamma$ for each $X_i \subset X$ such that for each $i$ with $X_i \subset A$ or $X_i \subset R$ it holds that $\gamma_i = e$ where $e$ is the neutral element in $\Gamma$. 
Let $Ind: X \to \N$ be the map which assigns to each $x \in X$ the corresponding index of the $X_i$ it is contained in.
\begin{Definition}
	 The map 
	 \begin{align*}
	 T_X:X \to X, x \mapsto \rho(\gamma_{Ind(x)})(x)
	 \end{align*}
	 is called the Turing map.
	 The group $\Gamma$ and the space $X$ (with all the choices of subsets and corresponding $\gamma_i$ made in the last paragraph) together with the Turing map will be called a Turing dynamical system and will be denoted by $(T_X)$
\end{Definition}
Let $x \in X$. If there is a $k \in \N$ such that $T_X^k(x) = T_X^{k + 1}(x)$ there also holds $T_X^h(x) = T_X^{h + 1}(x) \ \forall h \in \N$ with $h \geq k$. We then define $T_X^\infty (x) = T_X^k(x)$. Mostly we don't look at the map $T_X$ but the map $T_X^\infty$. We say that the Turing dynamical system $(T_X)$ accepts an input $y \in I$ if $T_X^\infty(y) \in A$ and it rejects it if $T_X^\infty(y) \in R$. In addition we say that $(T_X)$ holds for $y$ if it accepts or rejects it.

We can see that a Turing machine can be emulated with a Turing dynamical system. But first lets fix notation
\begin{Remark}
	Let $M \subset N^\Z$  be defined through $M = \{(n_i)_{i \in \Z} \in N^\Z | n_{-k} = m_{-k}, \cdots, n_l = m_l \} \ k,l \in \N$, i.e. a set where $k+l$ coordinates including the $0$-coordinate are fixed. We then simply write $[m_{-k} \cdots \underline{m_0} \cdots m_l]$ for $M$ and $[m_{-k} \cdots \underline{m_0} \cdots m_l][\sigma]$ for $M \times \{\sigma\}$.
\end{Remark}
\begin{Example}\label{TMtoTDS}
	Let  $T=(S,\delta, \tilde{A}, \tilde{R}, \tilde{I})$ be a Turing machine. We define $X = \2^\Z \times S$ and $\Gamma =   \2 \times \Z \times Bij(S)$.
	The action of $\Gamma$ on $X$ is defined by the following rules:
	\begin{itemize}
		\item The generator $\overline{1}$ of the group $\Z / 2\Z$ acts on the $\Z / 2\Z^\Z$ part of $X$ by adding $\overline{1}$ to the element with index $0$.
		\item The generator $1$ of the group $\Z$ acts also on the $\Z / 2\Z^\Z$ part by shifting every element 1 to the left, i.e. decreasing the index of every element by 1.
		\item $Bij(S)$ acts on $S$ in the natural way.
	\end{itemize}
	We will now use the transition function to construct the Turing map. We choose the following devision of $X$:
	\begin{align*}
	X = \bigcup_{x \in \2, \sigma \in S} [\underline{x}][\sigma]
	\end{align*}
	For an arbitrary $X_i = [\underline{x}][\sigma]$ we can use the transition function of the Turing machine and get $\delta(\sigma, x) = (\tilde{\sigma}, \alpha, \beta)$.\todo{sigma und x reihenfolge} Let $\tau \in Bij(S)$ be some map which sends $\sigma$ to $\tilde{\sigma}$ and let $y \in \2$ be $y=x-\alpha$. We can then see $y$ as a map from $\2$ to itself which sends $x$ to $\alpha$. Our element $\gamma_i$ corresponding to $X_i$ is then given by 
	\begin{align*}
	\gamma_i = \begin{cases}
	(id, \overline{0}, 0) & \text{if } X_i \subset A \cup R \\
	(y, \beta, \tau) & \text{else}
	\end{cases}
	\end{align*} 
	We only need to define $A$, $R$ and $I$. $A$ is given by
	\begin{align*}
	A = \bigcup_{x \in \2, \sigma \in \tilde{A}}[\underline{x}][\sigma]
	\end{align*}
	which is of course a union of some $X_i$. $R$ and $I$ can be defined analogous. The resulting Turing dynamical system emulates the Turing machine in the following way: 
	If we have an input $Y \in \2$ of the Turing machine we can transform it to an element $Y \times \tilde{I} \in X$. The Turing dynamical system accepts (rejects) $Y \times \tilde{I}$ exactly when the Turing machine accepts (rejects) $Y$.
\end{Example} 
\begin{Definition}
	The first fundamental set $\mathcal{F}_1(T_X)$ is the subset of $I$ consisting of those points $x$ with $T_X^\infty(x) \in A$ and there is no point $y$ with $T_X(y)=x$. The second fundamental set is the subset of $A$ defined as $\mathcal{F}_2(T_X)=T_X^\infty(\mathcal{F}_1(T_X))$. The first (second) fundamental value $\Omega_1(T_X) ( \Omega_2(T_X))$ is the measure of the corresponding fundamental set.
\end{Definition}

\begin{Definition}
	We say that a Turing dynamical system $(T_X)$
	\begin{itemize}
		\item stops at any configuration if $T_X^\infty (x) \in A \cup R$ for almost all $x \in X$
		\item has disjoint accepting chains if $T_X^\infty (x) \neq T_X^\infty (y)$ for almost all $x, y \in I$ with $x \neq y$
		\item does not restart if the set $T_X(X) \cap I$ has measure $0$
	\end{itemize}
\end{Definition}

\begin{Definition}
	Let $(T_X)$ be a Turing dynamical system where $X = \Pi_{j \in J} \2$ is an infinite product of $\2$ and each $X_i$  has only finitely many fixed elements, i.e. is of the form $X_i = \{(x_j)_{j \in J} \in X \ | \ x_k = 0 \ \forall k \in I_1, x_h = 1 \ \forall h \in I_2\, \ I_1, I_2 \subset J \ finite\}$. If in addition the action of $\Gamma$ is by continuous group automorphisms and $(T_X)$ stops at any configuration, has disjoint accepting chains and does not restart it is called a computing Touring dynamical system.
\end{Definition}
\todo{Warum cont group autos?}
\todo{maß}

Before we give an example of a computing Turing dynamical system we fix the notation of the shifting operation, because we will need it later on.
\begin{Definition} \label{shift}
	An element $z \in \Z$ acts on $\2^{\Z} = \prod_{i \in \Z} \2$ by decreasing every index by $z$. We call this action the shift action and denote it by $\zeta$. $\Z$ acts on $\bigoplus_{i \in \Z} \2$ in the same way which we will also call $\zeta$.
\end{Definition}
\begin{Example} \label{roTMtoTDS}
	Let $T$ be a read-only Turing machine. Let $(T_X)$ be the corresponding Turing dynamical system as constructed in \ref{TMtoTDS}. We assume that $T$ is in such a way that $(T_X)$ has disjoint accepting chains and stops at any configuration (which is not always the case). We will construct a computable Turing dynamical system from $T$.  We can easily assure that $(T_X)$ does not restart by adding a new state $I'$ to $T$ with $\delta(I', x) = \delta(I, x) \ \forall x \in \{0,1\}$ and setting $I'$ as the initial state of $T$. Because we do not change the symbols of the tape it suffices to use $\Gamma = \Z \times Bij(S)$ where $\Z$ operates on $X = \2^\Z \times S$ via shifting and $Bij(S)$ acts on $S$ in the natural way. Because a Turing machine has only finitely many states there exists a number $n \in \N$ such that $|S| < 2^n$. We identify every state with a different element $z \in \2^n$ with $z \neq 0$.
	
	 We can then assume that $X = \2^\Z \times \2^n$ by filling $\2^n$ with "dummy states" which will never be used. Because $Aut(\2^n)$ acts transitively on all of $\2^n / \{0\}$ we can assume that $\Gamma = \Z \times Aut(S)$. Because in the sets $X_i$ in \ref{TMtoTDS} there are only $n+1$ fixed components they are of the desired form and therefore the resulting Turing dynamical system is computing.
\end{Example}


\section{Computing $l^2$-Betti numbers}
In this chapter we want to prove the following central theorem which connects $l^2$-Betti numbers to the concept of Turing dynamical systems.
\begin{Theorem} \label{HS}
	Let $(T_X)$ be a computing Touring dynamical system. Then $\mu (I) - \Omega_1(T_X)$ is a $l^2$-Betti number arising from $\hat{X} \rtimes_{\hat{\rho}} \Gamma$ where $\hat{X}$ and $\hat{\rho}$ are the Pontryagin duals of the corresponding group or map.
\end{Theorem}
We will give the definition of Pontryagin duals and the semidirect product later in this chapter. The rest of this chapter is used to prove this theorem and give the needed definitions.
\subsection{Groupoids}
We begin by giving some algebraic definitions, namely the construct of groupoids which extends the definition of groups\todo{so sicher bin ich mir dabei jetzt net}. Therefore we require some basic knowledge about categories. \todo{Quelle zu Kategorientheorie}
\begin{Definition}
	A groupoid is a small category whose morphisms are all invertible.
\end{Definition}
\begin{Example} \label{group}
	Let $G$ be a group. We want to see how we can express $G$ as a groupoid. Let $\mathcal{G}_0 = \{\bullet\}$ be the set with only one element  and $\mathcal{G}$ be the category with objects $G_0$ and morphisms $G$ from $\bullet$ to $\bullet$.The composition of morphisms in $\mathcal{G}$ is the same as the multiplication in $G$. Then $\mathcal{G}$ is a small category and every morphism is invertible because every group element has an inverse. Therefore $\mathcal{G}$ is a groupoid.
\end{Example}
We will always denote the set of objects of a groupoid $\mathcal{G}$ by $\mathcal{G}_0$ and we can identify it with a subset of the set of morphisms of $\mathcal{G}$ by the embedding $\textbf{1}: \mathcal{G}_0 \to mor(\mathcal{G}), x \mapsto [id: x \to x]$ which sends every object to the corresponding identity morphism. Therefore we will only only look at $mor(\mathcal{G})$ and also call it $\mathcal{G}$. 

\begin{Definition}
	For every Groupoid $\mathcal{G}$ we define the maps $s: \mathcal{G} \to \mathcal{G}_0, [f:X \to Y] \mapsto X$ and $r: \mathcal{G} \to \mathcal{G}_0, [f:X \to Y] \mapsto Y$ which will be called source and range map.
\end{Definition}
\begin{Definition}
	A relation groupoid is a groupoid with with $|mor(X, Y)| \leq 1 \ \forall X,Y \in \G_0$.
\end{Definition}

\begin{Remark}
	if $X \in \G_0$ we denote by $\G X \subset \G_0$ the set of all those objects $Y \in \G_0$ with $mor(X, Y) \neq \emptyset$.
\end{Remark}
\todo{das mit den Pfeilen ist doof}
\begin{Definition}
	A discrete measurable groupoid is a groupoid where $\mathcal{G}$ is also a measurable space, $s, r$ and the maps gained through inverting or composition are all measurable and the fibers of $s$ and $r$ are countable.
	If in addition we have a measure $\mu$ such that 
	\begin{align*}
		\int_{\mathcal{G}_0} |r^{-1}(x) \cap U| d\mu(x) = \int_{\mathcal{G}_0} |s^{-1}(x) \cap U| d\mu(x) \\ \forall U \subset \mathcal{G}
	\end{align*}
	holds we call $\mathcal{G}$ discrete measured
\end{Definition}
\todo{Warum braucht man das eigentlich nochmal, was bedeutet das genau und stimmt das eigentlich? inverting map}
\begin{Definition}
	We say that a groupoid $\G$ has finite orbits if $\G X$ is finite for almost all $X \in \G_0$.
\end{Definition}
\begin{Definition}
	Let $\G$ be a relation groupoid with finite orbits. A measurable subset $D$ such that $|D \cap \G X| = 1$ for every $X \in \G$ with $\G X$ finite is called a fundamental domain.
\end{Definition}
\begin{Remark}
	One can show that every relation groupoid with finite orbits has a fundamental domain.
\end{Remark}
\todo{Beispiel oder Erklärung und wie das bei TDS ist}

\subsection{Groupoid ring}
A fundamental concept used for the calculation of $l^2$-Betti numbers of groups was the group ring. We will transfer this to the notion of groupoids.

\begin{Definition}
	Let $U$ be a subset of $\G_0$. A measurable edge is a map $\Phi:U \to \G$ such that $s \circ \Phi = id$ and $r \circ \Phi$ is injective.
\end{Definition}
\todo{Warum heißt das eigentlich messbar?}
From the definition we see, that defining a measurable edge means taking a subset of $\G_0$ and associating a morphism to every object such that the morphism starts in this object and no two such morphisms end in the same object. We want to define the inverse of a measurable edge in such a way, that it is also a measurable edge. Therefore it does not suffice to take the inverse of the map $\Phi$. Instead we invert every morphism in the image of $\Phi$.
\begin{Definition}
	Let $\Phi: U \to \G$ be a measurable edge. The inverse of $\Phi$ will be called $\Phi^{-1}$ and is defined as $\Phi^{-1}:r(Im(\Phi)) \to \G$ such that $\Phi^{-1}$ is a measurable edge and $\Phi^{-1} \circ r \circ \Phi (x) = \Phi (x)^{-1} \ \forall x \in r(Im(\Phi))$ where $\Phi (x)^{-1}$ is the inverse of the morphism $\Phi (x)$.
\end{Definition}
We now come to the definition of the groupoid ring. We want it to be ring of operators of $L^2(\G)$ \todo{Das sollte ich vorher mal irgendwann definiert haben}, such that it is in a way generated by measurable edges. For a measurable edge $\Phi$ we define a operator $\tilde \Phi$ through
\begin{align*}
	\tilde \Phi: L^2(\G) \to L^2(\G), F \mapsto \left[ \tilde \Phi (F): \G \to \C, \gamma \mapsto \begin{cases}
	F(\gamma \cdot_{\G} \Phi^{-1}(r(\gamma)) & \text{if } r(\gamma) \in Dom(\Phi^{-1}) \\
	0 & \text{otherwise}
	\end{cases} \right] .
\end{align*}
\todo{da sollte man ein Bild zu malen}
In addition for a map $f \in L^\infty (\G_0)$ we also define a operator $\tilde f$ on $L^2(\G)$ through
\begin{align*}
	\tilde f: L^2(\G) \to L^2(\G), F \mapsto [\tilde f(F): \G \to \C, \gamma \mapsto F(\gamma) \cdot f(r(\gamma))].
\end{align*}
\begin{Definition}
	For a groupoid $\G$ the groupoid ring $\C\G$ is the ring of bounded operators on $L^2(\G)$ generated by all measurable edges and all elements of $L^\infty(\G)$.
\end{Definition}
We always denote the elements of $\C\G$ by a linear combination $\sum_{i \in I} \tilde \Phi_i \cdot_{\C \G} \tilde f_i$ where $\Phi$ is a measurable edge, $f \in L^\infty(\G_0)$ and $I$ is a finite set. One can show that each element of $\C \G$ can be (although non-uniquely) represented in such a way. \todo{Evtl. könnte ich dazu noch ein bisschen was schreiben}
\begin{Example}
	Let $G$ be a group and $\G$ the corresponding groupoid as in Example \ref{group}. We want to show that the groupoid ring $\C \G$ is isomorphic to the g $\C G$. Because we have only one element in $\G_0$ the only measurable edges we have are the maps from $\bullet$ to a specific group element of $G$. The elements of $L^\infty(\G_0)$ are just maps from $\bullet$ to a specific element in $\C$. Therefore  $L^\infty(\G_0)$ is isomorphic to $\C$. Now let $\Phi_1: \bullet \mapsto g_1$ and $\Phi_2: \bullet \mapsto g_2$ be measurable edges. $\ldots$
\end{Example}
\todo{Das muss ich noch zuende machen}


\subsection{Groupoids of Turing dynamical systems}
\begin{Definition}
	Let $\Gamma$ be a discrete, countable group, $X$ a probability measure space and $\rho: \Gamma \to Bij(X)$ a right measure preserving action. The action groupoid $\G(\rho)$ is the groupoid with objects $X$ and morphisms $X \times \Gamma$ such that $s(x,\gamma) = x$ and $r(x, \gamma) = \rho(\gamma) (x)$. The composition of morphisms is defined through $(x, \gamma_1) \cdot_{\G(\rho)} (\rho(\gamma_1) (x), \gamma_2) = (x, \gamma_1 \cdot \gamma_2)$ and the inverse of $(x, \gamma)$ is $(\rho(\gamma) (x), \gamma^{-1})$.
\end{Definition}

\begin{Remark}
	Let $\G(\rho)$ be an action Groupoid of $\rho: \Gamma \to Bij(X)$. For each element $\gamma \in \Gamma$ we get a measurable edge $\bar{\gamma}: X \to x \times \Gamma$ which maps $x \in X$ to $(x, \gamma)$.
\end{Remark}

\begin{Remark}
	If $G$ is a locally compact Hausdorff topological group then there exists a unique normalized left Haar measure on $G$ which is in particular left translation invariant. Let $(T_X)$ be a computing Turing dynamical system. Then the Haar measure is defined on $X = \Pi_{j \in J} \2$. From now on every computing Turing dynamical system will always be equipped with the corresponding normalized Haar measure.
\end{Remark}
\todo{Quelle fuer Haar measure}
\begin{Lemma}
	Let $(T_X)$ be a computing Turing dynamical system with $X = \Pi_{j \in J} \2$ and equipped with the normalized Haar measure $\mu$. For a set $M = \{(x_j)_{j \in J} \in X \ | \ x_k = 0 \ \forall k \in I_1, x_h = 1 \ \forall h \in I_2\, \ I_1, I_2 \subset J \}$ it holds that
	\begin{align*}
		\mu (M) = \begin{cases}
		\frac{1}{2^{|I_1| \cdot |I_2|}} & if \ I_1 \ and \ I_2 \ are \ finite \\
		0 & if \ I_1 \ or \ I_2 \ is \ infinite
		\end{cases}.
	\end{align*}
\end{Lemma}
\begin{proof}
	Let $x \in X$ be an element with $\overline{0}$ on every coordinate $i \notin I_1 \cup I_2$. Then $xM$ is disjoint with $M$ and because of the left translation invariance of the measure it holds that $\mu (M) = \mu (xM)$. Because for every $i \in I_1 \cup I_2$ we can set the corresponding coordinate either to $\overline{1}$ or to $\overline{0}$ we get $2^{|I_1| \cdot |I_2|}$ of these sets if $I_1$ and $I_2$ are finite and an infinite amount of sets if one of them is infinite. The disjoint union of all of these sets is the whole set $X$ so the sum of these measures equals 1 and the claim follows.
\end{proof}
\begin{Remark}
	If $(T_X)$ is a Turing dynamical system with group action $\rho$, then we get a discrete measured groupoid through the action groupoid $\G(\rho)$.
\end{Remark}
\todo{warum discrete measured?}
\begin{Definition}
	Let $(T_X)$ be a Turing dynamical system with group action $\rho$. Then $\G (T_X) \subset \G(\rho)$ is the smallest groupoid with $\G (T_X)_0 = \G(\rho)_0$ and which contains all morphisms of the form $(x, \gamma_i)$ where $i = Ind(x)$ is the index of the corresponding $X_i$ that $x$ is contained in.
\end{Definition}
\begin{Lemma}
	$\G (T_X)$ is a relation groupoid with finite orbits for every computing Turing dynamical system $(T_X)$.
\end{Lemma}
\begin{proof}
	First let us have a look at $\G (T_X)$. We can generate it by setting $\G_0 = X$. We then add all morphisms of the form $(x, \gamma_{Ind(x)})$. We call this subset $\G'$. Of course $\G'$ is not a category in most cases. To ensure that $\G (T_X)$ is a groupoid we add a morphism from $x$ to $y$ for all $x, y \in \G_0$ where there already exist a path of morphisms from $x$ to $y$. In addition we add an identity morphism to every $ x\in \G_0$ which does not already have one and an inverse morphism to every morphism which does not already have one. 
	
	We can then see that we get one morphism from $x$ to $y$ in $\G (T_X)$ for every path of morphisms from $x$ to $y$ in $\G'$.But $x$ can have at maximum one outgoing morphism, therefore $\G (T_X)$ is a relation groupoid.
	
\end{proof}
\begin{Definition}
	Let $\G$ be a groupoid. The trace of an element $T \in \C \G$ in the groupoid ring is defined through
	\begin{align*}
		tr(T) = \langle T \chi_0, \chi_0 \rangle_{L^2 \G}
	\end{align*}
	where $\chi_0$ is the characteristic function of the objects $\G_0$ of $\G$.
\end{Definition}
\begin{Remark}
	Let $X \subset l^2G$ be an arbitrary subspace and $pr_X$ a projection onto $X$. Then $dim_{\mathcal{N}(G)}(X) = tr_{\mathcal{N}(G)}(pr_X)$.
\end{Remark}
\begin{Definition}
	Let $\G$ be a groupoid, $X \subset L^2 \G$ and $pr_X: L^2 \G \to X$ be projection onto $X$. The von-Neumann dimension of $X$ is defined through
	\begin{align*}
		dim_{\mathcal{N}(G)}(X) = tr_{\mathcal{N}(G)}(pr_X).
	\end{align*}
\end{Definition}
\subsection{Traces of Turing dynamical systems}
\begin{Theorem}
	Let $T_X$ be a computing Turing dynamical system with action $\rho$, division $\cup_i X_i$ with associated $\gamma_i$ and $\chi_i : \G \to \C$ be the characteristic function of $X_i$. Then for the element
	\begin{align*}
		S = (\sum_{i} \tilde{\bar{\gamma_i}} \tilde{\chi_i} \ + \chi_X - \chi_I - \chi_A - \chi_R)*(\sum_{i} \tilde{\bar{\gamma_i}} \tilde{\chi_i} \ + \chi_X - \chi_I - \chi_A - \chi_R) + \chi_A
	\end{align*}
	of the groupoid ring $\C \G(\rho)$ it holds that
	\begin{align*}
		dim_{\mathcal{N}(G)}(ker S) = \mu(I) - \Omega_1(T_X).
	\end{align*}
\end{Theorem}
Before we can proof this Theorem we have to do some preparations. 
\begin{Definition}
	Let $\G$ be a relation groupoid, $x \in \G_0$ and $T = \sum_{i \in I} \tilde \Phi_i\tilde f_i \in \C\G$. We then get an operator $T_x: l^2\G x \to l^2\G x$ which sends a basis element $1 \cdot y \in l^2\G x$ to $\sum_{i \in I'} r(\Phi_i(y)) f_i(y)$ where $I' \subset I$ is the set of indices where $y$ is in the domain of $\Phi_i$. 
\end{Definition}
\begin{Remark}
	If $\G$ is a relation groupoid with finite orbits then for almost all $x \in \G$ $l^2\G x$ is finite dimensional because $\G x$ is finite. Therefore $T_x$ has a finite dimensional kernel.
\end{Remark}
We will state the following fact without proof. For a complete proof see \cite{GRAB}.
\begin{Lemma}
	Let $\G$ be a relation groupoid with finite orbits and $D$ a fundamental domain of $\G$. Then for every $T \in \C \G$ it holds that
	\begin{align*}
		dim_{\mathcal{N}(G)}(ker T) = \int_D dim (ker T_x) \ d \mu (x).
	\end{align*} 
\end{Lemma}

\begin{Lemma}
	
\end{Lemma}
\subsection{Pontryagin duality}
We will now create a link between the groupoid ring and the group ring. For this we need the Pontryagin duals. For the rest of this chapter $X$ is a locally compact abelian group, $\Gamma$ is another group acting on $X$ by continuous group automorphisms. We will call this action $\rho: \Gamma \to Aut(X)$.
\begin{Definition}
	A character of $X$ is a homomorphism $\hat{x}: X \to S$, where $S$ is the multiplicative group of complex numbers of absolute value 1.The group of all characters of $X$ is called the character group or Pontryagin dual group of $X$ and is denoted by $\hat{X}$.
\end{Definition}
\todo{Beweis, dass das duale eine Gruppe ist?}
\begin{Definition}
	The Pontryagin dual $\hat{\rho}:\Gamma \to Aut(\hat{X})$ of the action $\rho$ is an action on $\hat{X}$ and defined as $\hat{\rho}(\gamma)(f)(x) = f(\rho(\gamma^{-1})(x))$ for $f \in \hat{X}$ and $x \in X$.
\end{Definition}
\todo{Um das ganze Auto-Zeug muss man sich mal noch Gedanken machen.}
\begin{Remark}
	An element of the Pontryagin dual $\hat{x} \in \hat{X}$ can also be seen as an element of $L^{\infty}(X)$. We call this element $P(x)$
\end{Remark}
\begin{Definition}
	Let $N, H$ be groups and $\Theta : H \to Aut(N)$ a homomorphism. The semidirect product $N \rtimes_\Theta H$ of $H$ and $N$ is defined on the Set $N \times H$ with multiplication 
	\begin{align*}
		(n_1, h_1) \cdot_{N \rtimes_\Theta H} (n_2, h_2) = (n_1 \cdot_N \Theta(h_1)(n_2), h_1 \cdot_H h_2).
	\end{align*}
\end{Definition}
%\begin{Remark}
%	Through the isomorphisms 
%	\begin{align*}
%		N \to N \rtimes_\Theta H, n \mapsto (n, e_H) \\ H \to N \rtimes_\Theta H, h \mapsto (e_N, h)
%	we can regard $N$ and $H$ as subgroups of $N \rtimes_\Theta H$. It then holds that $N \rtimes_\Theta H = NH$. We can therefore write $n \cdot_{N \rtimes_\Theta H} h$ for an element $(n,h)$.
%\end{Remark}
For more information about semidirect products see e.g. \cite{ALG}.
\begin{Theorem}
	The map 
	\begin{align*}
		P \otimes 1: \C(\hat{X} \rtimes_{\hat{\rho}} \Gamma) \to \C\G(\rho), \sum_I c_i \cdot (\hat{x_i}, \gamma_i) \mapsto \sum_I c_i \cdot \widetilde{P(\hat{x_i})} \cdot_{\C\G(\rho)} \widetilde{\bar{\gamma_i}} 
	\end{align*}
	 with $c_i \in \C$, $\hat{(x_i)} \in \hat{X}$ and $\gamma_i \in \Gamma$ is
	\begin{enumerate}
		\item a ringhomomorphism
		\item trace-preserving
		%\item injective
	\end{enumerate}
\end{Theorem}
\begin{proof}
	$i)$ Because of the definition it is clear that $P \otimes 1$ preserves the addition. So let $c_i \in \C$, $\hat{(x_i)} \in \hat{X}$ and $\gamma_i \in \Gamma$ with $i \in \{1,2\}$. It follows that
	\begin{align*}
		P \otimes 1(c_1 \cdot (\hat{x_1}, \gamma_1) \cdot_{C(\hat{X} \rtimes_{\hat{\rho}} \Gamma)} c_2 \cdot (\hat{x_2}, \gamma_2))  \\
		= P \otimes 1(c_1 \cdot  c_2 \cdot (\hat{x_1}, \gamma_1) \cdot_{N \rtimes_{\hat{\rho}} H} (\hat{x_2}, \gamma_2)) \\
		=  P \otimes 1(c_1 \cdot  c_2 \cdot (\hat{x_1} \cdot_{\hat{X}} \hat{\rho}(\gamma_1)(\hat{x_2}), \gamma_1 \cdot_{\Gamma} \gamma_2)) \\
		= c_1 \cdot  c_2 \cdot \widetilde{P(\hat{x_1} \cdot_{\hat{X}} \hat{\rho}(\gamma_1)(\hat{x_2}))} \cdot_{\C\G(\rho)} \widetilde{\overline{\gamma_1 \cdot_{\Gamma} \gamma_2}}.
	\end{align*}
	From the definition of the groupoid ring it follows directly that the last term equals 
	\begin{align*}
		c_1 \cdot  c_2 \cdot \widetilde{P(\hat{x_1})} \cdot_{\C\G(\rho)} \widetilde{P(\hat{\rho}(\gamma_1)(\hat{x_2}))} \cdot_{\C\G(\rho)} \widetilde{\bar{\gamma_1}} \cdot_{\C\G(\rho)} \widetilde{\bar{\gamma_2}}.
	\end{align*}
	Because elements of $\C$ commute with all elements in the groupoid ring it remains to show that 
	\begin{align*}
		\widetilde{P(\hat{\rho}(\gamma_1)(\hat{x_2}))} \cdot_{\C\G(\rho)} \widetilde{\bar{\gamma_1}} = \widetilde{\bar{\gamma_1}} \cdot_{\C\G(\rho)} \tilde{\hat{x_2}} 
	\end{align*}
	in the groupoid ring.
	Let $F \in L^2 \G(\rho)$ and $\alpha \in \G$. Then the left side equals
	\begin{align*}
		\widetilde{P(\hat{\rho}(\gamma_1)(\hat{x_2}))} \cdot_{\C\G(\rho)} \widetilde{\bar{\gamma_1}}(F)(\alpha) = \hat{x_2}(\rho (\gamma_1^{-1})( r(\alpha))) \cdot F(\alpha \cdot_{\G (\rho)} \bar{\gamma_1}^{-1}(r(\alpha)))  
	\end{align*}
	and the right side equals
	\begin{align*}
		\widetilde{\bar{\gamma_1}} \cdot_{\C\G(\rho)} \tilde{\hat{x_2}}(F)(\alpha) = \widetilde{\bar{\gamma_1}}(\hat{x_2}(r(\alpha)) \cdot F(\alpha)) = \hat{x_2}(r(\alpha \cdot_{\G (\rho)} \bar{\gamma_1}^{-1}(r(\alpha))) \cdot F(\alpha \cdot_{\G (\rho)} \bar{\gamma_1}^{-1}(r(\alpha)))
	\end{align*}
	which is the same.
	
	$ii)$ 
\end{proof}
\todo{braucht man *?, Elemente aus C an GR multiplizieren , mehr erklaeren?}

%Hier kommen die ganzen Sätze/Beweise usw.
\section{The lamplighter group}
\begin{Definition}
	The lamplighter group $L$ is defined as
	\begin{align*}
		L = (\bigoplus_{i \in \Z} \2) \rtimes_{\zeta} \Z.
	\end{align*}
	where $\zeta$ is the shift operation defined in \ref{shift}.
\end{Definition}
Lets first look at the Turing dynamical system corresponding to a read-only Turing machine as constructed in \ref{roTMtoTDS}. If we can assure that the resulting Turing dynamical system has disjoint accepting chains and stops at any configuration we can assure that it is computing. By using \ref{HS} we see that $\mu (I) - \Omega_1(T_X)$ is a $l^2$-Betti number arising from $\hat{X} \rtimes_{\hat{\rho}} \Gamma$. We then get
\begin{align*}
	\hat{X} \rtimes_{\hat{\rho}} \Gamma = (\widehat{(\prod_{i \in \Z} \2) \times \2^n}) \rtimes_{\hat{\rho}} (\Z \times Aut(\2^n)) \\
	= (\bigoplus_{i \in \Z} \2 \times \2^n) \rtimes_{\rho} (\Z \times Aut(\2^n)).
\end{align*}
Because $\Z$ acts only on the first part and $Aut(\2^n)$ only on the second part of $X$ and $\Z$ acts by shift we get
\begin{align*}
	\hat{X} \rtimes_{\hat{\rho}} \Gamma = ((\bigoplus_{i \in \Z} \2) \rtimes_{\zeta} \Z) \times (\2^n \rtimes_{\rho|_{Aut(\2^n)}} Aut(\2^n)) \\
	= L \times (\2^n \rtimes_{\rho|_{Aut(\2^n)}} Aut(\2^n)).
\end{align*}
Therefore we see that read-only Turing machines offer a way to compute the $l^2$-betti number of $L \times H$ for some discrete finite group $H$. We will use this fact to prove the following theorem.

\begin{Theorem} \label{mainTh}
	Every positive rational number arises from the lamplighter group.
\end{Theorem}
Lets start by constructing the used Turing dynamical system. 
\begin{Remark}
	In this chapter we will often describe Turing machines instead of Turing dynamical systems but because of \ref{roTMtoTDS} we can always construct the corresponding Turing dynamical system.
\end{Remark}
\begin{Lemma} \label{1TM}
	For every $m \in \N$ such that $\frac{1}{m}$ has a finite binary expansion there exists a computing Turing dynamical system $T_X$ with $\frac{\Omega_1(T_X)}{\mu(I)} = \frac{1}{m}$.
\end{Lemma}
\begin{proof}
	If $m = 1$ we can just construct a Turing machine which accepts every input. So let $m \neq 1$
	Let $0. a_1 a_2 \ldots a_k = \frac{1}{m}$ be the binary expansion of $\frac{1}{m}$. We construct a Turing machine $T$ with $k+3$ states. These states are divided into
	\begin{itemize}
		\item one accepting and one rejecting state called $s_A$ and $s_R$
		\item one initial state called $s_I$
		\item $k$ states called $s_1, s_2 \ldots s_k$.
	\end{itemize}
	The transition function is defined through 
	\begin{align*}
		\delta(s_I, x) = (s_1, x, 1) \ \forall x \in \2 \\
		\delta(s_i, \overline{0}) = \begin{cases}
			(s_{i+1}, \overline{0}, 1) & if \ i \neq k \\
			(s_R, \overline{0}, 0) & if \ i = k
		\end{cases} \\
		\delta(s_i, \overline{1}) = \begin{cases}
			(s_A, \overline{1}, 0) & if \ a_i = 1 \\
			(s_R, \overline{1}, 0) & if \ a_i = 0
		\end{cases}
	\end{align*}
	for all $i \in \{1 \ldots k\}$.	We now see that the Turing machine holds after at most $k+2$ steps independent of the current configuration. Therefore the resulting Turing dynamical system $(T_X)$ stops at any configuration. It is also clear that it does not restart. To ensure that $(T_X)$ has disjoint accepting chains we reduce the initial set $I$ of $(T_X)$ from $[\underline{x}][s_I]$ to $[\underline{\1}][s_I]$ (with $x \in \2$). Because $T$ holds at the first $\overline{1}$ it reads on the right side of its starting point we see that each configuration in the accepting or rejecting set of $(T_X)$ must be contained in
	\begin{align*}
		[\1 \ \0^n \ \underline{\1}][s] \ \text{with} \ s \in \{s_A, s_R\}, \ n \in \N \ \text{or} \ [\1 \ \0^{k-1} \ \underline{\0}][s_R].
	\end{align*} 
	Each of these elements can be traced back to a single element in
	\begin{align*}
	[\underline{\1} \ \0^{n} \ \1][s_I] \ \text{or} \ [\underline{\1}  \ \0^{k}][s_I].
	\end{align*} 
	where each entry which is not fixed stays the same. 
	It remains to show that $\frac{\Omega_1(T_X)}{\mu(I)} = \frac{1}{m}$. Let $K \subset \N$ be such that $a_i = 1 \Leftrightarrow i \in K$. It then holds that
	\begin{align*}
		\mathcal{F}_1(T_X) = \bigcup_{i \in K}[\underline{\1} \ \0^{i-1} \1][s_I].
	\end{align*}
	Let $n \in \N$ denote the number such that the space $X$ of $(T_X)$ equals $(\prod_{i \in \Z} \2) \times \2^n$, i.e. the smallest $n \in \N$ such that $k+3 < 2^n$. Then 
	\begin{align*}
		\mu([\underline{\1} \ \0^{i-1} \1][s_I]) = \frac{1}{2^n \cdot 2^ {i + 1}}
	\end{align*}
	and
	\begin{align*}
		\mu(I) = \mu([\underline{\1}][s_I]) = \frac{1}{2^{n + 1}}.
	\end{align*}
	Therefore 
	\begin{align*}
		\frac{\Omega_1(T_X)}{\mu(I)} = \sum_{i \in K} \frac{1}{2^i}
	\end{align*}
	which is exactly $\frac{1}{m}$.
\end{proof}

\begin{Lemma} \label{frac}
	For every $m \in \N$ with $m \neq 1$ the binary expansion of $\frac{1}{m}$ is of the form $0. a \overline{b}$ where $a$ is a nonrepeating and $b$ a repeating part. The length of $a$ and the length of $b$ are both finite.
\end{Lemma}
\todo{Kann man vllt. noch beweisen...}
\begin{Lemma} \label{2TM}
	For every $m \in \N$ there exists a computing Turing dynamical system $T_X$ with $\frac{\Omega_1(T_X)}{\mu(I)} = \frac{1}{m}$.
\end{Lemma}
\begin{proof}
	W.l.o.g. $m \neq 1$. Then with \ref{frac} follows that $\frac{1}{m} = 0.a_1a_2 \ldots a_k b_1 b_2 \ldots b_h$ where $a_1a_2 \ldots a_k$ is nonrepeating and $b_1 b_2 \ldots b_h$ is repeating. We then construct the same Turing machine as in \ref{1TM} but add $h$ many states called $r_1, r_2 \ldots r_h$. In addition we replace $\delta(s_k, \0) = (s_R, \0, 0)$ with $\delta(s_k, \0) = (r_1, \0, 1)$ and set
	\begin{align*}
		\delta(r_i, \overline{0}) = \begin{cases}
			(r_{i+1}, \overline{0}, 1) & if \ i \neq h \\
			(r_1, \overline{0}, 1) & if \ i = h
		\end{cases} \\
		\delta(r_i, \overline{1}) = \begin{cases}
			(s_A, \overline{1}, 0) & if \ b_i = 1 \\
			(s_R, \overline{1}, 0) & if \ b_i = 0
		\end{cases}
	\end{align*}
	which takes care of the repeating part of $\frac{1}{m}$. The rest of the transition function $\delta$ stays the same as in \ref{1TM}. We can see that this Turing machines does not hold for every input, because if we have only zeros on the right side of our staring point we never arrive at $s_A$ or $s_R$. But because the set $[\underline{x} \ \0 \ \0 \ldots][\sigma]$ (with $x \in \2$ and $\sigma$ an arbitrary state) has measure zero the resulting Turing dynamical system still stops at any configuration. The rest follows exactly like in the proof of \ref{1TM} because the infinite sum of measures converges.
\end{proof}

We can now start to proof \ref{mainTh}. Because of \ref{add} it suffices to show that for every $m \in \N$ $\frac{1}{m}$ arises from $L$. Let $m \in \N$  and $(T_X)$ be the Turing dynamical system constructed in the proof of \ref{2TM}. First we see that
\begin{align*}
	\mu (I) - \Omega_1(X) = \mu (I) (1 - \frac{\Omega_1(T_X)}{\mu(I)}) = \mu (I) (1 - \frac{1}{m})
\end{align*}
is a $l^2$-Betti number of $L \times (\2^n \rtimes_{\rho|_{Aut(\2^n)}} Aut(\2^n))$ for some $n \in \N$. Let $k$ denote the number of states of the Turing machine $T$ constructed in \ref{2TM} and $k' \in \N$ be such that $k < k'$ and $k' = 2^{h}$ for some $h \in \N$. We now change $(T_X)$ by adding "dummy states" until $(T_X)$ has $2^{k'}$ many states. Doing so does not change the value of $\frac{\Omega_1(T_X)}{\mu(I)}$. We now construct a Turing dynamical system $(T_{X'})$ in the same way as in \ref{roTMtoTDS} but we set $X' = \2^\Z \times \2^{k'}$ and identify every state of $T$ with an element of the standard basis of $\2^{k'}$. In addition we set $\Gamma = \Z \times Rot$ where $Rot$ is the subgroup of $Aut(\2^{k'})$ generated by the automorphism which rotates every element one to the right, i.e.
\begin{align*}
	\phi : \2^{k'} \to \2^{k'}, \sum_{i =1}^{k'}z_i \cdot e_i \mapsto \sum_{i =2}^{k'}z_{i - 1} \cdot e_i + z_{k'} \cdot e_1
\end{align*}
where $e_i$ denotes the standard basis of $\2^{k'}$ and each $z_i \in \2$. Then $Rot$ acts transitively on the standard basis of $\2^{k'}$ and has exactly $k' = 2^{h}$ elements. Then $\frac{1}{2^{k' + 1}} (1 - \frac{1}{m})$ is a $l^2$-Betti number of $L \times (\2^{k'} \rtimes_{\rho|_{Rot}} Rot)$. With \ref{mult} follows that 
\begin{align*}
	|(\2^{k'} \rtimes_{\rho|_{Rot}} Rot)| \cdot \frac{1}{2^{k' + 1}} (1 - \frac{1}{m}) = 2^{k'} \cdot 2^h \cdot \frac{1}{2^{k' + 1}} (1 - \frac{1}{m}) = 2^{h -1} (1 - \frac{1}{m})
\end{align*}
is a $l^2$-Betti number of $L$. By changing $I$ to
\begin{align*}
	[\1^{h-1} \underline{\1}][s_I]
\end{align*}
we get that $(1 - \frac{1}{m})$ is a $l^2$-Betti number of $L$. At last by interchanging $s_A$ and $s_R$ we get that $\frac{1}{m}$ is a $l^2$-Betti number of $L$ which concludes the proof.
  % Literaturverzeichnis (beginnt auf einer ungeraden Seite)
  \newpage

\begin{thebibliography}{Lam00}
 \bibitem{GRAB} Grabowski, \L ukasz: \emph{On Turing dynamical systems and the Atiyah problem}, Springer, 2014
 \bibitem{LUECK} Lück, Wolfgang $L^2$\emph{-Invariants: Theory and Applications to Geometry and K-Theory}, Springer, 2002
 \bibitem{HATCH} Hatcher, Allen \emph{Algebraic topology}, Cambridge University Press, 2015
 \bibitem{ALG} Aluffi, Paolo \emph{Algebra: Chapter 0}, in: Graduate Studies in Mathematics Volume 104, American Mathematical Society
\end{thebibliography}
 
      
  % ggf. hier Tabelle mit Symbolen 
  % (kann auch auf das Inhaltsverzeichnis folgen)

\newpage
  
 \thispagestyle{empty}


\vspace*{8cm}


\section*{Erklärung}

Hiermit versichere ich, dass ich diese Arbeit selbständig verfasst und keine anderen, als die angegebenen Quellen und Hilfsmittel benutzt, die wörtlich oder inhaltlich übernommenen Stellen als solche kenntlich gemacht und die Satzung des Karlsruher Instituts für Technologie zur Sicherung guter wissenschaftlicher Praxis in der jeweils gültigen Fassung beachtet habe. \\[2ex] 

\noindent
Ort, den Datum\\[5ex]

% Unterschrift (handgeschrieben)



\end{document}

